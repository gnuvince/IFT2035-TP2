\documentclass[10pt]{report}

\usepackage[utf8]{inputenc}
\usepackage[french]{babel}
\usepackage{amsmath}
\usepackage{amsfonts}
\usepackage{amssymb}
\usepackage{graphicx}
\usepackage{parskip}

\newcommand{\usage}[1]{\textbf{Utilisation: }\emph{#1}}

\begin{document}

\title{Rapport - Devoir 2}
\date{Novembre 2010}
\author{Vincent Foley-Bourgon (FOLV08078309) \and
  Eric Thivierge (THIE09016601)}

\maketitle

\section{Fonctions}

\subsection{Termes généraux}

Nous utiliserons le mot ``déballer'' pour signifier que nous appelons
une fonction dessinateur avec une fonction de transformation pour
obtenir la liste que le dessinateur contient.  Inversement, nous
utiliserons le mot ``emballer'' pour signifier que nous mettons une
liste dans une expression lambda prenant une fonction de
transformation comme argument.


\subsection{ligne}

\usage{(ligne depart arrivee)}

La fonction \emph{ligne} utilise la fonction utilitaire
\emph{liste-vects} afin de retourner une liste de points (vecteurs)
entre le vecteur de départ et le vecteur d'arrivée.
\emph{liste-vects} est une fonction partiellement terminale qui suit
la spécification donnée dans l'énoncé.

\subsection{parcours-$>$dessinateur}

\usage{(parcours-$>$dessinateur vecteurs)}

\emph{parcours-$>$dessinateur} prend une liste de vecteurs et retourne
un dessinateur qui joint chacun de ces vecteurs par une ligne (telle
que définie dans la sous-section précédente).  Les vecteurs sont pris
deux-à-deux, et les segment du dessinateur résultant sont concaténés
avec le \emph{parcours-$>$dessinateur} du reste des vecteurs.


\subsection{translation}

\usage{(translation dx dy dessinateur)}

La fonction de translation va déballer le dessinateur, appliquer
\emph{translate-segm} à chacun des segments du dessinateur, emballer
le résultat et retourner ce nouveau dessinateur.  La fonction
\emph{translate-segm} ajoute dx et dy aux coordonnées x et y des
vecteurs de départ et d'arrivée d'un segment.


\subsection{rotation}

\usage{(rotation angle dessinateur)}

La fonction de rotation va déballer le dessinateur, appliquer
\emph{rotate-segm} à chacun des segments du dessinateur, emballer le
résultat et retourner ce nouveau dessinateur.  La fonction
\emph{rotate-segm} effectue une rotation horaire de \emph{angle}
degrés aux vecteurs d'un segment.  Une fonction auxiliaire
\emph{degre-$>$radian} effecture la conversion de degrés à radians,
qui sont utilisés par les fonction trigonométrique de Scheme.

\subsection{reduction}

\usage{(reduction factx facty dessinateur)}

La fonction de réduction va déballer le dessinateur, appliquer
\emph{etire-segm} à chacun des segments du dessinateur, emballer le
résultat et retourner ce nouveau dessinateur.  La fonction
\emph{etire-segm} étire un segment par \emph{factx} horizontalement et
par \emph{facty} verticalement.




\end{document}
