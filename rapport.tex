\documentclass[10pt]{article}

\usepackage[utf8]{inputenc}
\usepackage[french]{babel}
\usepackage{amsmath}
\usepackage{amsfonts}
\usepackage{amssymb}
\usepackage{graphicx}
\usepackage{parskip}

\newcommand{\usage}[1]{\textbf{Utilisation: }\emph{#1}}

\begin{document}

\title{Rapport - Devoir 2}
\date{Novembre 2010}
\author{Vincent Foley-Bourgon (FOLV08078309) \and
  Eric Thivierge (THIE09016601)}

\maketitle

\section{Fonctions}

\subsection{Termes généraux}

Nous utiliserons le mot ``déballer'' pour signifier que nous appelons
une fonction dessinateur avec une fonction de transformation pour
obtenir la liste que le dessinateur contient.  Inversement, nous
utiliserons le mot ``emballer'' pour signifier que nous mettons une
liste dans une expression lambda prenant une fonction de
transformation comme argument.


\subsection{ligne}

\usage{(ligne depart arrivee)}

La fonction \emph{ligne} utilise la fonction utilitaire
\emph{liste-vects} afin de retourner une liste de points (vecteurs)
entre le vecteur de départ et le vecteur d'arrivée.
\emph{liste-vects} est une fonction partiellement terminale qui suit
la spécification donnée dans l'énoncé.

\subsection{parcours-$>$dessinateur}

\usage{(parcours-$>$dessinateur vecteurs)}

\emph{parcours-$>$dessinateur} prend une liste de vecteurs et retourne
un dessinateur qui joint chacun de ces vecteurs par une ligne (telle
que définie dans la sous-section précédente).  Les vecteurs sont pris
deux-à-deux, et les segment du dessinateur résultant sont concaténés
avec le \emph{parcours-$>$dessinateur} du reste des vecteurs.


\subsection{translation}

\usage{(translation dx dy dessinateur)}

La fonction de translation va déballer le dessinateur, appliquer
\emph{translate-segm} à chacun des segments du dessinateur, emballer
le résultat et retourner ce nouveau dessinateur.  La fonction
\emph{translate-segm} ajoute dx et dy aux coordonnées x et y des
vecteurs de départ et d'arrivée d'un segment.


\subsection{rotation}

\usage{(rotation angle dessinateur)}

La fonction de rotation va déballer le dessinateur, appliquer
\emph{rotate-segm} à chacun des segments du dessinateur, emballer le
résultat et retourner ce nouveau dessinateur.  La fonction
\emph{rotate-segm} effectue une rotation horaire de \emph{angle}
degrés aux vecteurs d'un segment.  Une fonction auxiliaire
\emph{degre-$>$radian} effectue la conversion de degrés à radians,
qui sont utilisés par les fonction trigonométrique de Scheme.


\subsection{loupe}

\usage{(loupe fact dessinateur)}

La fonction loupe déballe le dessinateur, applique la transformation
non-linéaire \emph{loupe-segm} à chacun des segments du
dessinateur. La fonction \emph{loupe-segm} transforme chaque vecteur
du segment par la fonction énoncée dans le devoir en utilisant
\emph{fact} en paramètre.

\subsection{reduction}

\usage{(reduction factx facty dessinateur)}

La fonction de réduction va déballer le dessinateur, appliquer
\emph{etire-segm} à chacun des segments du dessinateur, emballer le
résultat et retourner ce nouveau dessinateur.  La fonction
\emph{etire-segm} étire un segment par \emph{factx} horizontalement et
par \emph{facty} verticalement.

\subsection{superposition}

\usage{(superposition dessinateur1 dessinateur2)}

La fonction superposition déballe les deux dessinateurs, emballe leurs
segments respectifs et retourne le nouveau dessinateur ainsi créé.


\subsection{pile}

\usage{(pile prop dessinateur1 dessinateur2)}

La fonction pile déballe les deux dessinateurs, applique une réduction
verticale d'un facteur \emph{prop} sur le premier dessinateur et
\emph{1-prop} sur le deuxième. La fonction combine finalement les deux
ensembles de points déplacés verticalement dans les mêmes proportions.


\subsection{cote-a-cote}

\usage{(cote-a-cote prop dessinateur1 dessinateur2)}

La fonction cote-a-cote déballe les deux dessinateurs, applique une
réduction horizontale d'un facteur \emph{prop} sur le premier
dessinateur et \emph{1-prop} sur le deuxième. La fonction combine
finalement les deux ensembles de points déplacés horizontalement dans
les mêmes proportions.


\subsection{entier-$>$dessinateur}

\usage{(entier-$>$dessinateur n)}

La fonction entier-$>$dessinateur commence par extraire les chiffres
composant le nombre \emph{n}. La fonction construit ensuite la
représentation du nombre en appelant récursivement la fonction
\emph{cote-a-cote} sur les chiffres réduits horizontalement en
proportion inverse du nombre de chiffres dans \emph{n}.


\subsection{arbre-$>$dessinateur}

\usage{(arbre-$>$dessinateur lst)}

Cette fonction va dessiner un arbre binaire.  Elle fait appel à
plusieurs fonctions auxiliaires que nous allons examiner:

\begin{itemize}
\item \emph{profondeur-arbre}: cette fonction prend un arbre et
  retourne sa profondeur en parcourant récursivement le sous-arbre de
  gauche et le sous-arbre de droite.  On utilise la profondeur de
  l'arbre pour fixer la hauteur de chacun des niveaux de l'arbre quand
  on le dessinera.
\item \emph{arbre-compte-feuilles}: cette fonction prend un arbre et
  retourne un arbre (quasi-) isomorphe qui contient sur chaque noeud
  interne le nombre de feuilles dans ce sous-arbre.  Cette information
  sera utilisée pour espacer horizontalement les branches.  On utilise
  un arbre, car autrement on devrait recalculer à chaque fois le
  nombre de feuilles, ce qui causerait une perte inutile de
  performance.
\item \emph{arbre-$>$dessinateur-aux}: la fonction qui va dessiner
  l'arbre; elle prend 6 paramètres: l'arbre à dessiner, l'arbre
  contenant le nombre de feuilles, la hauteur d'un niveau et les
  positions x et y de la racine.  Cette fonction s'appelle
  récursivement jusqu'à ce qu'elle reçoive un arbre vide, à quel
  moment elle retourne le dessinateur vide.  Si on a pas un arbre
  vide, on utilise les fonctions \emph{superposition} et \emph{ligne}
  pour dessiner les deux branches.
\end{itemize}




\section{Impressions de Scheme}

\subsection{Vincent}

\subsubsection{Points positifs de Scheme}

\begin{itemize}
\item Les fonctions anonymes et les fermetures permettent de
  facilement paramétrer des fonctions et permettent des abstractions
  qui seraient impossibles ou inconvenantes en Java.
\item Les fonctions d'ordre supérieur rendent plus concis la
  manipulation de listes.
\item L'optimisation des appels terminaux permet d'écrire des
  fonctions sans effets de bord, mais dont les performances sont
  équivalentes à celles d'un langage impératif.
\item L'utilisation d'une liste pour stocker les segments permet une
  manipulation facile d'un dessin via les fonctions d'ordre supérieur
  (ex.: \emph{map}).
\item Le ``garbage-collector'' permet de manipuler des structures
  complexes sans avoir à se soucier de les libérer nous-mêmes.
\item On peut accomplir beaucoup de choses complexes sans jamais avoir
  à modifier le contenu d'une variable.
\item L'intégration de l'interpréteur \emph{gsi} avec Emacs permet un
  cycle de développement beaucoup plus rapide que ce qui est possible
  avec des langages comme C ou Java.
\item Les fonctions \emph{trace} et \emph{time} sont très utiles pour
  analyser et comparer la performance d'une fonction.
\end{itemize}

\subsubsection{Points négatifs de Scheme}
\begin{itemize}
\item L'absence de typage statique permet à des bogues -- qui seraient
  normalement capturés à la compilation -- de ne se manifester qu'à
  l'exécution et seulement si on a la bonne fortune de tester avec les
  bons (ou mauvais) paramètres.
\item L'interface et l'implantation des structures de données n'est
  pas assez séparée.  L'utilisation de la paire comme base pour nos
  types, ainsi que l'absence de typage statique, fait que la
  responsabilité de s'assurer que toutes les données sont bien
  construites repose entièrement sur les épaules du programmeur.  De
  plus, des données créées avec \emph{cons} et \emph{list} plutôt que
  \emph{vect} et \emph{segm} fontionneraient, mais briseraient si on
  décidait de modifier la représentation des dessins (ex.: utiliser
  des vecteurs plutôt que des listes).
\item L'utilisation de la paire comme base pour les arbres leur donne
  un ``goût'' très ad hoc.  On peut par exemple décider de mettre
  l'information propre à un noeud dans le \emph{car} de la paire et de
  mettre les deux branches dans une autre paire qu'on stock dans le
  \emph{cdr} de la paire-noeud.  Évidemment, la responsabilité de
  respecter ce choix de design est celle du programmeur et le
  compilateur ne peut pas aider.
\item Il n'est pas très logique d'avoir des fonctions qui retournent
  des valeurs de types différents (ex.: \emph{assoc}).  Une meilleure
  solution serait d'établir une norme (étant donné qu'on ne encoder
  cette décision dans le système de type et laisser le compilateur
  faire la police) dictant que si une méthode ne retourne ``rien'',
  elle retourne la liste vide, et que si elle retourne une valeur,
  elle la retourne dans une liste. \footnote{Ceci donnerait quelque
    chose de similaire au type \emph{option} de ML.}
\item La syntaxe préfixe est un peu lourde dans les calculs
  mathématiques.
\item Dans Emacs, il n'y a aucune complétion de code, détection
  d'erreurs de syntaxe ou d'accès à la documentation comme on peut
  trouver dans des IDE pour des langages comme Java.  Peut-être qu'une
  intégration de Gambit avec SLIME pourrait aider?
\item Scheme ne possède pas de ``pattern matching'' (comme dans ML ou
  Haskell) qui permet de ``déconstruire'' une structure de données.
  Cela demande donc de faire beaucoup d'appels manuels de fonction
  pour accéder aux membres internes d'une structure de données (voir
  les appels à \emph{caaddr} dans \emph{arbre-$>$dessinateur-aux}).
\end{itemize}

\subsection{Eric}

\subsubsection{Forces de Scheme pour cette application}

\begin{itemize}
\item Une bonne stratégie pour développer cette application est de la
  séparer en une suite de couches d'abstraction de représentation des
  objets composant un dessin (du dessin physique à l'écran, la
  représentation des vecteurs et des segments, jusqu'aux dessinateurs
  et fonctions de transformation des dessins). Scheme possède (au
  moins) deux avantages en ce sens: (1) L'utilisation du typage
  dynamique facilite la mise en oeuvre de cette stratégie. La
  représentation des données est déterminée localement ou au niveau
  d'abstraction inférieur et ne dépend que d'un contrat fonctionnel
  implicite entre ces deux niveaux. L'utilisation de fonctions
  abstraites de création et de contrôle (vect, vect?, vect-x etc)
  garantie le respect de ces contrats. (2) L'utilisation des
  fermetures permet une manipulation fonctionnelle facile et efficace
  des objets abstraits de haut niveau (dessinateur, arbre etc).
\item L'application développée dans ce travail est fondamentalement un
  problème mathématique de représentation de formes géométriques et de
  graphes. L'utilisation du caractère fonctionnel de Scheme est donc
  tout à fait naturel dans ce cas. Il y a une très petite ``distance''
  entre les concepts mathématiques et leur représentation dans le
  langage Scheme. La quantité de travail mise dans l'analyse
  conceptuelle mathématique du problème pour mettre en évidence la
  nature du problème contribue proportionnellement (grande efficience)
  à la résolution du problème. De plus la compréhension des fondements
  mathématiques du problème peut, en général, nous donner des pistes
  d'optimisation du programme.
\item La récupération automatique des structures de données mortes
  (via le GC) libère le programmeur de cette tâche et le code de
  toutes les instructions sans relation avec le problème.
\item L'utilisation de Java nous aurait obligé une quantité de travail
  non négligeable pour bâtir les structures de données. Scheme réduit
  la complexité de la représentation des données en utilisant la
  syntaxe du langage comme structure de données.
\item En Java, les niveaux d'abstraction auraient été implantés par
  les classes, ce qui est une façon assez efficace de le faire mais
  ceci nous aurait éloigné conceptuellement de la nature mathématique
  du problème.
\end{itemize}

\subsubsection{Faiblesses de Scheme pour cette application}
\begin{itemize}
\item Le développement dans un langage à typage dynamique impose au
  développeur toute la responsabilité de vérifier le type des données
  passées aux fonctions. Ceci est un problème relativement mineur pour
  un petit projet comme ce devoir ou pour un produit mature où le
  contrat de chaque fonction est bien défini et documenté. Cette
  caractéristique peut compliquer la tâche du programmeur dans le
  contexte d'un produit non mature pour lequel les structures de
  données et les relations fonctionnelles évoluent beaucoup.
\item L'utilisation de Java aurait permis de puiser dans un grand
  nombre de librairies existantes et possiblement optimisées pour nos
  besoins.

\end{itemize}

\subsection{Environnements de développement}
Nous n'avons utilisé, pour réaliser ce travail, que l'environnement
Emacs. Emacs est un excellent outil de développement pour les langages
de la famille Lisp. L'utilisation d'un IDE tel Jazz/Jedi aurait
peut-être ajouté à notre productivité. Dans les faits, son
implantation dans mon environnement de travail (MacOS X) s'est avérée
difficile et a donc été abandonnée.


\section{Représentation fonctionnelle}

Utiliser des fonctions pour représenter des dessins présentent
certains avantages: les dessins sont facilement composables (voir la
fonction \emph{compose}) et en passant une fonction de transformation
on peut facilement paramétrer un dessin.

Quelques autres avantages:

\begin{itemize}
\item Les fonctions récursives terminales permettent de garder la
  solution computationnelle très proche de la nature mathématique du
  problème sans pour autant sacrifier la performance de l'application.
\item Dans le cadre d'un projet de grande envergure, la programmation
  fonctionnelle permettrait de paralléliser les calculs d'un grand
  nombre de dessin.
\end{itemize}

La représentation fonctionnelle présente également des inconvénients:
les performances sont moins bonnes que si on utilisait un
représentation matricielle.  Par exemple, si on compose des appels à
\emph{rotation}, \emph{translation} et \emph{reduction}, on va:

\begin{enumerate}
\item Déballer le dessin
\item Appliquer une transformation à chaque vecteur de chaque segment
  du dessin
\item Ré-emballer le dessin
\end{enumerate}

Et ce processus sera répété trois fois pour chaque appel.  Dans une
représentation matricielle, comme ces opérations sont linéaires, on
pourrait multiplier les trois matrices les décrivant ensemble et
appliquer la matrice résultante une seule fois aux vecteurs.  Pour de
grands dessins, on aurait un gain de performance appréciable.




\end{document}
